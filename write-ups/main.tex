\documentclass[a4paper,12pt]{article}
\usepackage[english]{babel}
\usepackage[left=2.5cm,right=2.5cm,top=3cm,bottom=2.5cm]{geometry}
%\usepackage{mathtools}
%\usepackage{amsthm}
%\usepackage{amsmath}
%\usepackage{nccmath}
%\usepackage{amssymb}
%\usepackage{amsfonts}
%\usepackage{physics}
%\usepackage{dsfont}
%\usepackage{mathrsfs}

%\usepackage{minted}

\usepackage{titling}
\usepackage{indentfirst}

%\usepackage{bm}
%\usepackage[dvipsnames]{xcolor}
\usepackage[x11names]{xcolor}
%\usepackage{cancel}

\usepackage{xurl}
\usepackage[colorlinks=true]{hyperref}

\usepackage{float}
\usepackage{graphicx}
%\usepackage{tikz}

%%%%%%%%%%%%%%%%%%%%%%%%%%%%%%%%%%%%%%%%%%%%%%%%%%%

\newcommand{\eps}{\epsilon}
\newcommand{\vphi}{\varphi}

\newcommand{\N}{{\mathbb{N}}}
\newcommand{\Z}{{\mathbb{Z}}}
\newcommand{\Q}{{\mathbb{Q}}}
\newcommand{\R}{{\mathbb{R}}}
\newcommand{\C}{{\mathbb{C}}}
\renewcommand{\S}{{\hat{S}}}

\newcommand{\cc}[1]{\overline{#1}}
\newcommand{\unit}[1]{\; \mathrm{#1}}

\newcommand{\n}{\medskip}
\newcommand{\e}{\quad \mathrm{and} \quad}
\newcommand{\ou}{\quad \mathrm{or} \quad}
\newcommand{\ptodo}{\forall \,}
\newcommand{\implies}{\; \Rightarrow \;}
%\newcommand{\eqname}[1]{\tag*{#1}} % Tag equation with name

\setlength{\droptitle}{-7em}


%\definecolor{LightGray}{gray}{0.9}  % requires xcolor package
%\usepackage{minted}

% https://tex.stackexchange.com/questions/173850/problem-in-adding-a-background-color-in-a-minted-environment
\usepackage{tcolorbox}
\tcbuselibrary{minted,skins}
\newtcblisting{bash}[1][]{
  listing engine=minted,
  colback=bashcodebg,
  colframe=black!70,
  listing only,
  minted style=colorful,
  minted language=bash,
  minted options={linenos=true,numbersep=3mm,texcl=true,#1},
  left=5mm,enhanced,
  overlay={\begin{tcbclipinterior}\fill[black!25] (frame.south west)
            rectangle ([xshift=5mm]frame.north west);\end{tcbclipinterior}}
}
\definecolor{bashcodebg}{rgb}{0.85,0.85,0.85}

\newcommand{\bashinline}[1]{%
\colorbox{bashcodebg}{%
\parbox[b][0.6em]{\widthof{\mintinline{coq}{#1}}}{\mintinline{coq}{#1}}%
}%
}

\title{\Huge{\textbf{CyberSec}}}
\author{Mateus Marques}

\begin{document}

\maketitle

\section{Linux Fundamentals}

Command \bashinline{uname -a} to discover which Linux distribution you are in.

\n

The \bashinline{PS1} variable can be modified to display useful information, such as full hostname (IP addresses in some cases) and date.

\n

To find help about Linux, have in mind \bashinline{man} and \bashinline{apropos}.

\n

Commands that display system information:
\begin{bash}
whoami, id, hostname, hostnamectl (systemd), uname, pwd, ifconfig,
ip, netstat, ss, ps, who, env, lsblk, lsusb, lsof, lspci.
\end{bash}

The command \bashinline{uname -r} can be useful, because we can search for a kernel specific version exploit. Example: search for ``4.15.0-99-generic exploit'', and the first \href{https://www.exploit-db.com/exploits/47163}{result} immediately appears useful to us.

\n

Connect to VPN (\bashinline{sudo} is necessary):
\begin{bash}
sudo openvpn academy-regular.ovpn
\end{bash}

To connect via SSH:
\begin{bash}
ssh [username]@[IP address]
\end{bash}

The command \bashinline{stat} displays metadata about files, for example its inode (index number), birth, last modification, etc.

\n

The \bashinline{tree} command is much more useful than listing one directory by one with \bashinline{ls}.

\n

Commands for file searching: \bashinline{which, find, locate}.

\n

What is the name of the config file that has been created after \bashinline{2020-03-03} and is smaller than \bashinline{28k} but larger than \bashinline{25k}?
\begin{bash}
find / -type f -name *.conf -size +25k -size -28k \
-exec ls -la {} \; 2>/dev/null
\end{bash}

If we try to use the argument \bashinline{-newerBt 2020-03-03} we get
``\texttt{find: This system does not provide a way to find the birth time of a file.}''

\n

How many files exist on the system that have the \bashinline{".bak"} extension?
\begin{bash}
find / -type f -name *.bak 2>/dev/null | wc -l
\end{bash}

\end{document}
